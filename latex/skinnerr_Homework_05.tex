\documentclass[11pt]{article}

\input{../../Latex_Common/skinnerr_latex_preamble_asen5417.tex}

%%
%% DOCUMENT START
%%

\begin{document}

\pagestyle{fancyplain}
\lhead{}
\chead{}
\rhead{}
\lfoot{\hrule ASEN 5417: Homework 5}
\cfoot{\hrule \thepage}
\rfoot{\hrule Ryan Skinner}

\noindent
{\Large Homework 5}
\hfill
{\large Ryan Skinner}
\\[0.5ex]
{\large ASEN 5417: Numerical Methods}
\hfill
{\large Due 2015/10/20}\\
\hrule
\vspace{6pt}

%%%%%%%%%%%%%%%%%%%%%%%%%%%%%%%%%%%%%%%%%%%%%%%%%
%%%%%%%%%%%%%%%%%%%%%%%%%%%%%%%%%%%%%%%%%%%%%%%%%
\section{Introduction} %%%%%%%%%%%%%%%%%%%%%%%%%%
%%%%%%%%%%%%%%%%%%%%%%%%%%%%%%%%%%%%%%%%%%%%%%%%%
%%%%%%%%%%%%%%%%%%%%%%%%%%%%%%%%%%%%%%%%%%%%%%%%%

\subsection{Problem 1}

The partial differential equation (PDE) that governs one-dimensional heat transfer, with constants and boundary conditions given for a 2 cm-thick steel pipe, is
\begin{equation}
\frac{\partial u}{\partial t} = \frac{k}{c \rho} \frac{\partial^2 u}{\partial x^2}
\;,
\qquad
\begin{aligned}
u(0,t) &= 0 \;, \\
u(2,t) &= 0 \;, \\
u(x,0) &= 100 sin(\pi x / 2) \;,
\end{aligned}
\qquad
\begin{aligned}
k &= 0.13 \; &&\text{cal / sec cm \degree C} \;, \\
c &= 0.11 \; &&\text{cal / g \degree C} \;, \\
\rho &= 7.8 \; &&\text{g / cm$^3$}
\;,
\end{aligned}
\label{eq:prob1}
\end{equation}
where $u(x,t)$ is the temperature in \degree C. Numerically integrate this PDE using the explicit forward-time, central-space (FTCS, or Euler) method with $\Delta x = 0.1$ cm. Choose $\Delta t$ such that the diffusion criterion is (a) $d = 0.5$ and (b) $d = 1.0$. Compare your solution for (a) at $t = \{1, 2, 4, 8\}$ sec with the analytical solution
\begin{equation}
u = 100 \exp(-0.3738t) \sin(\pi x / 2)
\;.
\label{eq:prob1_analytic}
\end{equation}
Note that for (b), the diffusion criterion is higher than the one allowed by the von Neumann method. For this case, plot your solutions at approximately $t = \{0.1, 0.5, 1, 2\}$ sec, and comment on stability.

\subsection{Problem 2}

Solve the system from Problem 1 using the implicit Crank-Nicolson method with forward time differencing. The Thomas algorithm can be utilized for this purpose. Try values for the diffusion criterion of $d = \{0.5, 1, 10\}$, and comment on the solution's stability.

\subsection{Problem 3}

Solve the system from Problem 1, but with an adiabatic boundary condition at the upper boundary,
\begin{equation}
u(0,t) = 0 \;, \qquad \frac{\partial u}{\partial x}(2,t) = 0
\;.
\label{eq:prob3}
\end{equation}

%%%%%%%%%%%%%%%%%%%%%%%%%%%%%%%%%%%%%%%%%%%%%%%%%
%%%%%%%%%%%%%%%%%%%%%%%%%%%%%%%%%%%%%%%%%%%%%%%%%
\section{Methodology} %%%%%%%%%%%%%%%%%%%%%%%%%%%
%%%%%%%%%%%%%%%%%%%%%%%%%%%%%%%%%%%%%%%%%%%%%%%%%
%%%%%%%%%%%%%%%%%%%%%%%%%%%%%%%%%%%%%%%%%%%%%%%%%

\subsection{Problem 1}

Discretizing \eqref{eq:prob1} using the forward-time, central-space (FTCS) method with explicit time advancement yields
\begin{equation}
\frac{u_i^{n+1} - u_i^n}{\Delta t} = \alpha \frac{u_{i+1}^n - 2u_i^n + u_{i-1}^n}{\Delta x^2}
\;,
\end{equation}
where the subscript denotes the spatial grid point, and the superscript indexes the time step. Further note it is convenient to define the coefficient of thermal diffusivity as $\alpha \equiv k / c \rho = 0.1515$ m/s$^2$. Solving for the unknown quantity, we obtain
\begin{equation}
u_i^{n+1} = d u_{i-1}^n + (1-2d) u_i^n + d u_{i+1}^n
\;,
\label{eq:ftcs}
\end{equation}
where the diffusion number is defined as
\begin{equation}
d = \frac{\alpha \Delta t}{\Delta x^2}
\;.
\end{equation}
Using \eqref{eq:ftcs}, it is trivial to step forward in time with Dirichlet boundary conditions on $u$.

To obtain the requested diffusion numbers, we set (a) $\Delta t = 0.033$ and (b) $\Delta t = 0.066$.

\subsection{Problem 2}

\subsection{Problem 3}

%%%%%%%%%%%%%%%%%%%%%%%%%%%%%%%%%%%%%%%%%%%%%%%%%
%%%%%%%%%%%%%%%%%%%%%%%%%%%%%%%%%%%%%%%%%%%%%%%%%
\section{Results} %%%%%%%%%%%%%%%%%%%%%%%%%%%%%%%
%%%%%%%%%%%%%%%%%%%%%%%%%%%%%%%%%%%%%%%%%%%%%%%%%
%%%%%%%%%%%%%%%%%%%%%%%%%%%%%%%%%%%%%%%%%%%%%%%%%

\subsection{Problem 1}

\begin{figure}[h!]
\begin{center}
\includegraphics[height=1.95in]{Prob1a_u.eps}
\includegraphics[height=1.95in]{Prob1a_err.eps}
\\[-0.5cm]
\caption{FTCS solutions to \eqref{eq:prob1} compared to analytical solutions \eqref{eq:prob1_analytic}, and the point-wise relative error.}
\label{fig:Prob2}
\end{center}
\end{figure}

\subsection{Problem 2}

\subsection{Problem 3}

%%%%%%%%%%%%%%%%%%%%%%%%%%%%%%%%%%%%%%%%%%%%%%%%%
%%%%%%%%%%%%%%%%%%%%%%%%%%%%%%%%%%%%%%%%%%%%%%%%%
\section{Discussion} %%%%%%%%%%%%%%%%%%%%%%%%%%%%
%%%%%%%%%%%%%%%%%%%%%%%%%%%%%%%%%%%%%%%%%%%%%%%%%
%%%%%%%%%%%%%%%%%%%%%%%%%%%%%%%%%%%%%%%%%%%%%%%%%

\subsection{Problem 1}

\subsection{Problem 2}

\subsection{Problem 3}

%%%%%%%%%%%%%%%%%%%%%%%%%%%%%%%%%%%%%%%%%%%%%%%%%
%%%%%%%%%%%%%%%%%%%%%%%%%%%%%%%%%%%%%%%%%%%%%%%%%
\section{References} %%%%%%%%%%%%%%%%%%%%%%%%%%%%
%%%%%%%%%%%%%%%%%%%%%%%%%%%%%%%%%%%%%%%%%%%%%%%%%
%%%%%%%%%%%%%%%%%%%%%%%%%%%%%%%%%%%%%%%%%%%%%%%%%

No external references were used other than the course notes for this assignment.

%%%%%%%%%%%%%%%%%%%%%%%%%%%%%%%%%%%%%%%%%%%%%%%%%
%%%%%%%%%%%%%%%%%%%%%%%%%%%%%%%%%%%%%%%%%%%%%%%%%
\section*{Appendix: MATLAB Code} %%%%%%%%%%%%%%%%
%%%%%%%%%%%%%%%%%%%%%%%%%%%%%%%%%%%%%%%%%%%%%%%%%
%%%%%%%%%%%%%%%%%%%%%%%%%%%%%%%%%%%%%%%%%%%%%%%%%

The following code listings generate all figures presented in this homework assignment.

%\includecode{Problem_1.m}
%\includecode{Problem_2.m}

%%
%% DOCUMENT END
%%
\end{document}
