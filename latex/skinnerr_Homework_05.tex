\documentclass[11pt]{article}

\input{../../Latex_Common/skinnerr_latex_preamble_asen5417.tex}

%%
%% DOCUMENT START
%%

\begin{document}

\pagestyle{fancyplain}
\lhead{}
\chead{}
\rhead{}
\lfoot{\hrule ASEN 5417: Homework 5}
\cfoot{\hrule \thepage}
\rfoot{\hrule Ryan Skinner}

\noindent
{\Large Homework 5}
\hfill
{\large Ryan Skinner}
\\[0.5ex]
{\large ASEN 5417: Numerical Methods}
\hfill
{\large Due 2015/10/20}\\
\hrule
\vspace{6pt}

%%%%%%%%%%%%%%%%%%%%%%%%%%%%%%%%%%%%%%%%%%%%%%%%%
%%%%%%%%%%%%%%%%%%%%%%%%%%%%%%%%%%%%%%%%%%%%%%%%%
\section{Introduction} %%%%%%%%%%%%%%%%%%%%%%%%%%
%%%%%%%%%%%%%%%%%%%%%%%%%%%%%%%%%%%%%%%%%%%%%%%%%
%%%%%%%%%%%%%%%%%%%%%%%%%%%%%%%%%%%%%%%%%%%%%%%%%

\subsection{Problem 1}

The partial differential equation (PDE) that governs one-dimensional heat transfer, with constants and boundary conditions given for a 2 cm-thick steel pipe, is
\begin{equation}
\frac{\partial u}{\partial t} = \frac{k}{c \rho} \frac{\partial^2 u}{\partial x^2}
\;,
\qquad
\begin{aligned}
u(0,t) &= 0 \;, \\
u(2,t) &= 0 \;, \\
u(x,0) &= 100 sin(\pi x / 2) \;,
\end{aligned}
\qquad
\begin{aligned}
k &= 0.13 \; &&\text{cal / sec cm \degree C} \;, \\
c &= 0.11 \; &&\text{cal / g \degree C} \;, \\
\rho &= 7.8 \; &&\text{g / cm$^3$}
\;,
\end{aligned}
\label{eq:prob1}
\end{equation}
where $u(x,t)$ is the temperature in \degree C. Numerically integrate this PDE using the explicit forward-time, central-space (FTCS, or Euler) method with $\Delta x = 0.1$ cm. Choose $\Delta t$ such that the diffusion criterion is (a) $d = \tfrac{1}{2}$ and (b) $d = 1$. Compare your solution for (a) at $t = \{1, 2, 4, 8\}$ sec with the analytical solution
\begin{equation}
u = 100 \exp(-0.3738t) \sin(\pi x / 2)
\;.
\label{eq:prob1_analytic}
\end{equation}
Note that for (b), the diffusion criterion is higher than the one allowed by the von Neumann method. For this case, plot your solutions at approximately $t = \{\tfrac{1}{10}, \tfrac{1}{2}, 1, 2\}$ sec, and comment on stability.

\subsection{Problem 2}

Solve the system from Problem 1 using the implicit Crank-Nicolson method with forward time differencing. The Thomas algorithm can be utilized for this purpose. Try values for the diffusion criterion of $d = \{\tfrac{1}{2}, 1, 10\}$, and comment on the solution's stability.

\subsection{Problem 3}

Solve the system from Problem 1, but with an adiabatic boundary condition at the upper boundary,
\begin{equation}
u(0,t) = 0 \;, \qquad \frac{\partial u}{\partial x}(2,t) = 0
\;.
\label{eq:prob3}
\end{equation}

%%%%%%%%%%%%%%%%%%%%%%%%%%%%%%%%%%%%%%%%%%%%%%%%%
%%%%%%%%%%%%%%%%%%%%%%%%%%%%%%%%%%%%%%%%%%%%%%%%%
\section{Methodology} %%%%%%%%%%%%%%%%%%%%%%%%%%%
%%%%%%%%%%%%%%%%%%%%%%%%%%%%%%%%%%%%%%%%%%%%%%%%%
%%%%%%%%%%%%%%%%%%%%%%%%%%%%%%%%%%%%%%%%%%%%%%%%%

\subsection{Problem 1}

Discretizing \eqref{eq:prob1} using the forward-time, central-space (FTCS) method with explicit time advancement yields
\begin{equation}
\frac{u_i^{n+1} - u_i^n}{\Delta t} = \alpha \frac{u_{i+1}^n - 2u_i^n + u_{i-1}^n}{\Delta x^2}
\;,
\end{equation}
where the subscript denotes the spatial grid point, and the superscript indexes the time step. Further note it is convenient to define the coefficient of thermal diffusivity as $\alpha \equiv k / c \rho = 0.1515$ m/s$^2$. Solving for the unknown quantity, we obtain
\begin{equation}
u_i^{n+1} = d u_{i-1}^n + (1-2d) u_i^n + d u_{i+1}^n
\;,
\label{eq:ftcs}
\end{equation}
where the diffusion number is defined as
\begin{equation}
d = \frac{\alpha \Delta t}{\Delta x^2} 
\quad
\rightarrow
\quad
\Delta t = \frac{d \Delta x^2}{\alpha}
\;.
\label{eq:diffusion_number}
\end{equation}
Using \eqref{eq:ftcs}, it is trivial to step forward in time with Dirichlet boundary conditions on $u$. We use \eqref{eq:diffusion_number} to obtain the requested diffusion numbers by setting (a) $\Delta t = 0.033$ and (b) $\Delta t = 0.066$.

\subsection{Problem 2}

The implicit Crank-Nicolson method with forward time-differencing is most-commonly used to solve the diffusion equation, which can be written as
\begin{equation}
\frac{u_i^{n+1} - u_i^n}{\Delta t}
=
\frac{\alpha}{2 \Delta x^2}
\left(
u_{i+1}^{n+1} - 2 u_i^{n+1} + u_{i-1}^{n+1} + u_{i+1}^n - 2 u_i^n + u_{i-1}^n
\right)
\;.
\end{equation}
It is a simple matter to re-write this as a linear equation relating the value of $u_i$ at the next time step to its spatial neighbors at the next time step, and its spatial neighbors at the current time step:
\begin{equation}
u_i^{n+1}
=
\beta
\left(
u_{i+1}^{n+1} - 2 u_i^{n+1} + u_{i-1}^{n+1} + u_{i+1}^n - 2 u_i^n + u_{i-1}^n
\right)
+ u_i^n
\end{equation}
\begin{equation}
  \underbrace{\left(      -\beta \right)}_{b_i} u_{i-1}^{n+1}
+ \underbrace{\left( 1 + 2 \beta \right)}_{a_i} u_i^{n+1}
+ \underbrace{\left(      -\beta \right)}_{c_i} u_{i+1}^{n+1}
=
\underbrace{\beta \left( u_{i+1}^n - 2 u_i^n + u_{i-1}^n \right) + u_i^n}_{d_i}
\;,
\end{equation}
where
\begin{equation}
\beta = \frac{\alpha \Delta t}{2 \Delta x^2}
\;,
\end{equation}
is known, along with all point-wise values of $u^n$. We solve the system of equations using Dirichlet boundary conditions with the same method detailed in Homework 3.

\subsection{Problem 3}

%%%%%%%%%%%%%%%%%%%%%%%%%%%%%%%%%%%%%%%%%%%%%%%%%
%%%%%%%%%%%%%%%%%%%%%%%%%%%%%%%%%%%%%%%%%%%%%%%%%
\section{Results} %%%%%%%%%%%%%%%%%%%%%%%%%%%%%%%
%%%%%%%%%%%%%%%%%%%%%%%%%%%%%%%%%%%%%%%%%%%%%%%%%
%%%%%%%%%%%%%%%%%%%%%%%%%%%%%%%%%%%%%%%%%%%%%%%%%

\subsection{Problem 1}

Results for the FTCS method are presented in \figref{fig:Prob1}.

\begin{figure}[h!]
\begin{center}
\includegraphics[scale=0.62]{Prob1a_err.eps}
\hspace*{-0.2cm}
\includegraphics[scale=0.62]{Prob1a_u.eps}
\\
\includegraphics[scale=0.62]{Prob1b_err.eps}
\hspace*{-0.2cm}
\includegraphics[scale=0.62]{Prob1b_u.eps}
\\[-0.5cm]
\caption{FTCS solutions to \eqref{eq:prob1} compared to analytical solutions \eqref{eq:prob1_analytic}, and the point-wise relative error over time. Upper plots are for $d=\tfrac{1}{2}$ ($\Delta t = 0.033$), and lower plots correspond to $d=1$ ($\Delta t = 0.066$).}
\label{fig:Prob1}
\end{center}
\end{figure}

\subsection{Problem 2}

Results for the Crank-Nicolson method with Dirichlet boundary conditions are presented in \figref{fig:Prob2}.

\begin{figure}[h!]
\begin{center}
\includegraphics[scale=0.62]{Prob2_d0p5_err.eps}
\hspace*{-0.2cm}
\includegraphics[scale=0.62]{Prob2_d0p5_u.eps}
\\
\includegraphics[scale=0.62]{Prob2_d1_err.eps}
\hspace*{-0.2cm}
\includegraphics[scale=0.62]{Prob2_d1_u.eps}
\\
\includegraphics[scale=0.62]{Prob2_d10_err.eps}
\hspace*{-0.2cm}
\includegraphics[scale=0.62]{Prob2_d10_u.eps}
\\[-0.5cm]
\caption{Crank-Nicolson solutions to \eqref{eq:prob1} compared to analytical solutions \eqref{eq:prob1_analytic}, and the point-wise relative error over time. Diffusion numbers $d$ are annotated. For $d = 10$, $\Delta t = 0.6601$.}
\label{fig:Prob2}
\end{center}
\end{figure}

\subsection{Problem 3}

%%%%%%%%%%%%%%%%%%%%%%%%%%%%%%%%%%%%%%%%%%%%%%%%%
%%%%%%%%%%%%%%%%%%%%%%%%%%%%%%%%%%%%%%%%%%%%%%%%%
\section{Discussion} %%%%%%%%%%%%%%%%%%%%%%%%%%%%
%%%%%%%%%%%%%%%%%%%%%%%%%%%%%%%%%%%%%%%%%%%%%%%%%
%%%%%%%%%%%%%%%%%%%%%%%%%%%%%%%%%%%%%%%%%%%%%%%%%

\subsection{Problem 1}

As can be seen in \figref{fig:Prob1}, FTCS results for a diffusion number of $d=\tfrac{1}{2}$ have fairly low relative error at all time-steps considered, and no numerical instability is present. For a condition number of $d=1$, which should be unstable according to the von Neumann method, we see high-frequency instabilities manifest at $t \sim 2.0$. Relative error grows rapidly, as can be seen at $t \sim 2.1$. Later times are not shown, because the solution quickly becomes non-physical.

\subsection{Problem 2}

As shown in \figref{fig:Prob2}, the Crank-Nicolson method produces results with point-wise errors slightly better than the FTCS method, at diffusion numbers at least $10\times$ greater than the maximum stable diffusion number of the FTCS method. We conclude that the Crank-Nicolson method is very stable. Though we do need to solve a matrix system at each time step, $\Delta t$ can be much higher than the FTCS method, and thus the Crank-Nicolson method has the potential for much higher efficiency.

\subsection{Problem 3}

%%%%%%%%%%%%%%%%%%%%%%%%%%%%%%%%%%%%%%%%%%%%%%%%%
%%%%%%%%%%%%%%%%%%%%%%%%%%%%%%%%%%%%%%%%%%%%%%%%%
\section{References} %%%%%%%%%%%%%%%%%%%%%%%%%%%%
%%%%%%%%%%%%%%%%%%%%%%%%%%%%%%%%%%%%%%%%%%%%%%%%%
%%%%%%%%%%%%%%%%%%%%%%%%%%%%%%%%%%%%%%%%%%%%%%%%%

No external references were used other than the course notes for this assignment.

%%%%%%%%%%%%%%%%%%%%%%%%%%%%%%%%%%%%%%%%%%%%%%%%%
%%%%%%%%%%%%%%%%%%%%%%%%%%%%%%%%%%%%%%%%%%%%%%%%%
\section*{Appendix: MATLAB Code} %%%%%%%%%%%%%%%%
%%%%%%%%%%%%%%%%%%%%%%%%%%%%%%%%%%%%%%%%%%%%%%%%%
%%%%%%%%%%%%%%%%%%%%%%%%%%%%%%%%%%%%%%%%%%%%%%%%%

The following code listings generate all figures presented in this homework assignment.

%\includecode{Problem_1.m}
%\includecode{Problem_2.m}

%%
%% DOCUMENT END
%%
\end{document}
